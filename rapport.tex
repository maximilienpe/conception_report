\documentclass{scrreprt}
\usepackage{array}
\usepackage{graphicx}
\usepackage{listings}
\usepackage{underscore}
\usepackage[bookmarks=true]{hyperref}
\usepackage[utf8]{inputenc}
\usepackage{float}
\usepackage[french]{babel}
\hypersetup{
    bookmarks=false,    % show bookmarks bar
    pdftitle={rapport_conception_Lambolez_Petit},    % title
    pdfauthor={Théodore Lambolez, Maximilien Petit},                     % author
    pdfsubject={TeX and LaTeX},                        % subject of the document
    pdfkeywords={TeX, LaTeX, graphics, images}, % list of keywords
    colorlinks=true,       % false: boxed links; true: colored links
    linkcolor=blue,       % color of internal links
    citecolor=black,       % color of links to bibliography
    filecolor=black,        % color of file links
    urlcolor=black,        % color of external links
    linktoc=page            % only page is linked
}
\def\myversion{1.0}
\date{}
%\title
\usepackage{hyperref}
\begin{document}
\begin{figure}
   \begin{minipage}[c]{.46\linewidth}
      \includegraphics[scale=0.3]{images/telecom.png}
   \end{minipage} \hfill
   \begin{minipage}[c]{.46\linewidth}
      \includegraphics[scale=1.9]{images/lorraine.jpg}
   \end{minipage}
\end{figure}
\begin{flushright}
    \rule{15cm}{5pt}
    \vskip1cm
\end{flushright}
\begin{center}
	\vspace{3cm}
	\fbox{
	\begin{minipage}{0.9\textwidth}
        	\Huge{
			\textbf{
			\begin{center}
				Projet de conception : \\
				Gestion des Stages
				\vspace{0.5cm}
			\end{center}
			}
		}
	\end{minipage}
	}
\end{center}
\begin{flushright}
        \vspace{5cm}
	\huge{
        \textbf{
	Ecrit par \\
	\vspace{0,875cm}
	\href{mailto:theodore.lambolez@telecomnancy.eu}{Théodore Lambolez} \\
	\href{mailto:maximilien.petit@telecomnancy.eu}{Maximilien Petit}\\
	}
	}
        \vspace{0,5cm}
        \large{
	\textbf{
	\today\\
	}	
	}
\end{flushright}

\tableofcontents

\chapter{Introduction}
%\addcontentsline{toc}{chapter}{Introduction}

Ce devoir de conception permet l'approfondissement du travail déjà réalisé lors du 
cahier des charges. Cette partie a pour but de rendre plus efficace et plus réfléchie la phase 
de développement du logiciel. Ici, nous ne nous intéresserons pas à tous les cas d'utilisation 
de la gestion des stages mais seulement à certains d'entre eux dans le but de nous exercer. Ainsi, 
nous détaillerons à l'aide de diagrammes UML le système de saisie de la fiche de renseignement, de  
validation de cette fiche ainsi que le système de passage d'une année à l'autre.  

	Pour ce faire, nous verrons d'abord le diagramme d'activité qui nous aidera à avoir une vision 
d'ensemble de la demande de stage jusqu'à la signature de la convention en passant par le remplissage
de la fiche de renseignement. Ensuite, nous approfondirons les trois digrammes de séquences. Le diagramme
de classe placé en dernière partie poura etre utilisé comme annexe. 

\newpage
\chapter{Diagramme d'activité}
%\addcontentsline{toc}{chapter}{Diagramme d'activité}

	Le processus de demande de stage étant assez complexe. Nous avons choisi de le représenter
en deux étapes. La première étape est spécifique à la réalisation de la fiche de renseignement. La 
seconde étape est quant à elle spécifique à la réalisation de la convention. 

	La fiche de renseignement est élaborée en deux temps. L'élève commence par fournir quelques
renseignements tels que le nom de l'entreprise et une adresse mail qui servira à créer un compte pour 
cette entreprise dans le cas où elle n'en n'aurait pas déjà. Un mail de création de compte sera envoyé
automatiquement à l'adresse précisée par l'élève. Ainsi, le responsable du stage en entreprise pourra
valider son mail et commencer à saisir les informations supplémentaires concernant l'entreprise et 
le contenu du stage. Une fois envoyées, ces informations seront analysées par le responsable de stage
en université. Deux choix suivent. Soit la fiche de renseignement est assez satisfaisante pour être 
validée, soit elle ne l'est pas. Dans le dernier cas, à nouveau deux choix s'offrent au professeur 
responsable. Soit il décide que l'entreprise ne peut pas satisfaire aux exigeances du stage et 
réinitialise tout le processus de demande, soit il décide qu'il faudrait davantage d'information pour valider
la fiche et envoie une notification contenant possible une remarque au responsable en entreprise
qui aura de nouveau accès au formulaire de renseignement. 

%\begin{figure}[!h]
%\centering
%\includegraphics[width=15cm]{images/firstStepActivityDiagramme.png}
%\caption{Première étape du diagramme d'activité}
%\end{figure}

\newpage
	La création de la convention s'effectue de même en deux moments. On commencera par la réaliser 
automatiquement ou manuellement si la première solution est insatisfaisante. La phase de signature des trois
partis suit directement. On procédera dans l'ordre de signature suivant : professeur responsable, élève 
et enfin responsable en entreprise. Finalement, l'administration récupèrera la convention complètement signée.

%\begin{figure}[!h]
%\centering
%\includegraphics[width=15cm]{images/secondStepActivityDiagramme.png}
%\caption{Première étape du diagramme d'activité}
%\end{figure}

\newpage
\chapter{Diagramme de séquence}
%\addcontentsline{toc}{chapter}{Diagramme de séquence}

\section{Cas de la saisie}

\section{Cas de la validation}

\section{Cas du chabgement d'année}



\newpage
\chapter{Diagramme de classe}
%\addcontentsline{toc}{chapter}{Diagramme de classe}





\end{document}
