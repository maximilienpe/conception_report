\documentclass{scrreprt}
\usepackage{array}
\usepackage{graphicx}
\usepackage{listings}
\usepackage{underscore}
\usepackage[bookmarks=true]{hyperref}
\usepackage[utf8]{inputenc}
\usepackage{float}
\usepackage[french]{babel}
\hypersetup{
    bookmarks=false,    % show bookmarks bar
    pdftitle={rapport_conception_Lambolez_Petit},    % title
    pdfauthor={Théodore Lambolez, Maximilien Petit},                     % author
    pdfsubject={TeX and LaTeX},                        % subject of the document
    pdfkeywords={TeX, LaTeX, graphics, images}, % list of keywords
    colorlinks=true,       % false: boxed links; true: colored links
    linkcolor=blue,       % color of internal links
    citecolor=black,       % color of links to bibliography
    filecolor=black,        % color of file links
    urlcolor=black,        % color of external links
    linktoc=page            % only page is linked
}
\def\myversion{1.0}
\date{}
%\title
\usepackage{hyperref}
\begin{document}
\begin{figure}
   \begin{minipage}[c]{.46\linewidth}
      \includegraphics[scale=0.3]{images/telecom.png}
   \end{minipage} \hfill
   \begin{minipage}[c]{.46\linewidth}
      \includegraphics[scale=1.9]{images/lorraine.jpg}
   \end{minipage}
\end{figure}
\begin{flushright}
    \rule{15cm}{5pt}
    \vskip1cm
\end{flushright}
\begin{center}
	\vspace{3cm}
	\fbox{
	\begin{minipage}{0.9\textwidth}
        	\Huge{
			\textbf{
			\begin{center}
				Projet de conception : \\
				Gestion des Stages
				\vspace{0.5cm}
			\end{center}
			}
		}
	\end{minipage}
	}
\end{center}
\begin{flushright}
        \vspace{5cm}
	\huge{
        \textbf{
	Ecrit par \\
	\vspace{0,875cm}
	\href{mailto:theodore.lambolez@telecomnancy.eu}{Théodore Lambolez} \\
	\href{mailto:maximilien.petit@telecomnancy.eu}{Maximilien Petit}\\
	}
	}
        \vspace{0,5cm}
        \large{
	\textbf{
	\today\\
	}	
	}
\end{flushright}

\tableofcontents

\chapter{Introduction}
\addcontentsline{toc}{chapter}{Introduction}

	Ce devoir de conception permet l'approfondissement du travail déjà réalisé lors du
cahier des charges. Cette partie a pour but de rendre plus efficace et plus réfléchie la phase
de développement du logiciel. Ici, nous ne nous intéresserons pas à tous les cas d'utilisation
de la gestion des stages mais seulement à certains d'entre eux dans le but de nous exercer. Ainsi,
nous détaillerons à l'aide de diagrammes UML le système de saisie de la fiche de renseignement, de validation de cette fiche ainsi que le système de passage d'une année à une autre.  

\newpage
\chapter{Diagramme de classe}
\addcontentsline{toc}{chapter}{Diagramme de classe}



\newpage
\chapter{Diagramme de séquence}
\addcontentsline{toc}{chapter}{Diagramme de séquence}

\section{Cas de la saisie}

\section{Cas de la validation}

\section{Cas du chabgement d'année}



\newpage
\chapter{Diagramme d'activité}
\addcontentsline{toc}{chapter}{Diagramme d'activité}




\end{document}

