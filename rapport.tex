\documentclass{scrreprt}
\usepackage{array}
\usepackage{graphicx}
\usepackage{listings}
\usepackage{underscore}
\usepackage[bookmarks=true]{hyperref}
\usepackage[utf8]{inputenc}
\usepackage{float}
\usepackage[french]{babel}
\hypersetup{
    bookmarks=false,    % show bookmarks bar
    pdftitle={rapport_conception_Lambolez_Petit},    % title
    pdfauthor={Théodore Lambolez, Maximilien Petit},                     % author
    pdfsubject={TeX and LaTeX},                        % subject of the document
    pdfkeywords={TeX, LaTeX, graphics, images}, % list of keywords
    colorlinks=true,       % false: boxed links; true: colored links
    linkcolor=blue,       % color of internal links
    citecolor=black,       % color of links to bibliography
    filecolor=black,        % color of file links
    urlcolor=black,        % color of external links
    linktoc=page            % only page is linked
}
\def\myversion{1.0}
\date{}
%\title
\usepackage{hyperref}
\begin{document}
\begin{figure}
   \begin{minipage}[c]{.46\linewidth}
      \includegraphics[scale=0.3]{images/telecom.png}
   \end{minipage} \hfill
   \begin{minipage}[c]{.46\linewidth}
      \includegraphics[scale=1.9]{images/lorraine.jpg}
   \end{minipage}
\end{figure}
\begin{flushright}
    \rule{15cm}{5pt}
    \vskip1cm
\end{flushright}
\begin{center}
	\vspace{3cm}
	\fbox{
	\begin{minipage}{0.9\textwidth}
        	\Huge{
			\textbf{
			\begin{center}
				Projet de conception : \\
				Gestion des Stages
				\vspace{0.5cm}
			\end{center}
			}
		}
	\end{minipage}
	}
\end{center}
\begin{flushright}
        \vspace{5cm}
	\huge{
        \textbf{
	Ecrit par \\
	\vspace{0,875cm}
	\href{mailto:theodore.lambolez@telecomnancy.eu}{Théodore Lambolez} \\
	\href{mailto:maximilien.petit@telecomnancy.eu}{Maximilien Petit}\\
	}
	}
        \vspace{0,5cm}
        \large{
	\textbf{
	\today\\
	}	
	}
\end{flushright}

\tableofcontents

\chapter{Introduction}
%\addcontentsline{toc}{chapter}{Introduction}

Ce devoir de conception permet l'approfondissement du travail déjà réalisé lors du 
cahier des charges. Cette partie a pour but de rendre plus efficace et plus réfléchie la phase 
de développement du logiciel. Ici, nous ne nous intéresserons pas à tous les cas d'utilisation 
de la gestion des stages mais seulement à certains d'entre eux dans le but de nous exercer. Ainsi, 
nous détaillerons à l'aide de diagrammes UML le système de saisie de la fiche de renseignement, de  
validation de cette fiche ainsi que le système de passage d'une année à l'autre.  

	Pour ce faire, nous verrons d'abord le diagramme d'activité qui nous aidera à avoir une vision 
d'ensemble de la demande de stage jusqu'à la signature de la convention en passant par le remplissage
de la fiche de renseignement. Ensuite, nous approfondirons les trois digrammes de séquences. Le diagramme
de classe placé en dernière partie poura etre utilisé comme annexe. 

\newpage
\chapter{Diagramme d'activité}
%\addcontentsline{toc}{chapter}{Diagramme d'activité}

	Le processus de demande de stage étant assez complexe. Nous avons choisi de le représenter
en deux étapes. La première étape est spécifique à la réalisation de la fiche de renseignement. La 
seconde étape est quant à elle spécifique à la réalisation de la convention. 

	La fiche de renseignement est élaborée en deux temps. L'élève commence par fournir quelques
renseignements tels que le nom de l'entreprise et une adresse mail qui servira à créer un compte pour 
cette entreprise dans le cas où elle n'en n'aurait pas déjà. Un mail de création de compte sera envoyé
automatiquement à l'adresse précisée par l'élève. Ainsi, le responsable du stage en entreprise pourra
valider son mail et commencer à saisir les informations supplémentaires concernant l'entreprise et 
le contenu du stage. Une fois envoyées, ces informations seront analysées par le responsable de stage
en université. Deux choix suivent. Soit la fiche de renseignement est assez satisfaisante pour être 
validée, soit elle ne l'est pas. Dans le dernier cas, à nouveau deux choix s'offrent au professeur 
responsable. Soit il décide que l'entreprise ne peut pas satisfaire aux exigeances du stage et 
réinitialise tout le processus de demande, soit il décide qu'il faudrait davantage d'information pour valider
la fiche et envoie une notification contenant possible une remarque au responsable en entreprise
qui aura de nouveau accès au formulaire de renseignement. 

%\begin{figure}[h]
%\centering
%\includegraphics[width=15cm]{images/firstStepActivityDiagramme.png}
%\caption{Première étape du diagramme d'activité}
%\end{figure}

\newpage
	La création de la convention s'effectue de même en deux moments. On commencera par la réaliser 
automatiquement ou manuellement si la première solution est insatisfaisante. La phase de signature des trois
partis suit directement. On procédera dans l'ordre de signature suivant : professeur responsable, élève 
et enfin responsable en entreprise. Finalement, l'administration récupèrera la convention complètement signée.

%\begin{figure}[h]
%\centering
%\includegraphics[width=15cm]{images/secondStepActivityDiagramme.png}
%\caption{Seconde étape du diagramme d'activité}
%\end{figure}

\newpage
\chapter{Diagramme de séquence}
%\addcontentsline{toc}{chapter}{Diagramme de séquence}

	Dans cette section nous aborderons les trois diagrammes de séquences demandés : la saisie de la fiche de renseignement,
la validation de celle-ci par le professeur responsable du stage ainsi que la transition d'une année scolaire à une autre.

\newpage
\section{Cas de la saisie}

	Ce diagramme de séquence sera traité en deux diagrammes différents puisque notre choix de conception est d'avoir réalisé
la fiche de renseignement en deux saisies distinctes.

	Tout d'abord une fois qu'un étudiant aura rempli le formulaire de la première partie de la fiche de renseignement en ligne et aura
appuyé sur le bouton "soumettre les informations", la classe gérant l'interface graphique appellera sur le contrôleur la fonction 
submitStudentData(Data d). La dénomination "Data d" fait en fait ici référent à tous les renseignements que l'élève donne lors de 
cette saisie. Les plus importantes pour bien comprendre ici sont le nom de l'entreprise et le mail du futur référent du stage au sein
celle-ci. Suite à ceci, le contrôleur gère la suite. Il commencera par déterminer si l'entreprise existe déjà ou non. Dans le cas où
ce teste s'avère être négatif, il créera lui-même une instance de la classe "Company" en remplissant les premières informations la 
concernant comme son nom. Ensuite, le contrôleur devras rechercher l'instance de la classe "Student" correspondant à l'étudiant qui
a effectué la saisie, de manière à ce que le contrôleur puisse créer une instance de la classe "InformationSheet" qui aura accès aux instances
de "Student" et de "Company" qui la concerne. Ce nouvel objet stage sera créé en initialisant certaines de ses données avec celles qui lui
auront été founies. De plus, cette instance de "InformationSheet" s'ajoutera aux instance de "Student" et "Company" correspondant pour finir 
d'effectuer le lien.

\newpage
\begin{figure}[h]
\centering
\includegraphics[width=15cm]{images/submitStudentSeqDiagram.png}
\caption{Diagramme de séquence de la saisie de la fiche de renseignement par l'étudiant}
\end{figure}

	Pour ce qui est du diagramme de séquence plus modeste, concernant la saisie de la fiche de renseignement par l'entreprise, 
la seule action réalisée est par le contrôleur lorsqu'il recoit les informations à travers la fonction "submitCompanyInformation(Data d)",
est de compléter les informations incluses dans la bonne instance de "InformationSheet".

\begin{figure}[h]
\centering
\includegraphics[width=15cm]{images/submitCompanySeqDiagram.png}
\caption{Diagramme de séquence de la saisie de la fiche de renseignement par le responsable en entreprise}
\end{figure}

\newpage
\section{Cas de la validation}

	Le diagramme de séquence représentant le cas de la validation d'une fiche de renseignement par le professeur référent, débute 
à partir du moment où la classe gérant l'interface graphique appelle la fonction "submitDecision(Data d)". On suppose ici que 
le professeur a eu les choix suivant : valider la fiche de renseignement, fiche de renseignement invalide puisque ne correspondant 
pas aux exigeances du stage ou invalide car elle est incomplète. 

	Dans le cas où la fiche est bien validée, on se propose de sauvegarder dans la bonne instance de "InformationSheet" le fait qu'elle est 
validée. Ensuite, puisqu'on est certain d'avoir une bonne version de la fiche de renseignement, on sauvegarde les informations en dur 
dans la base de donnée. Enfin, on envoie un mail ou une simple notification applicative dans le but de prévenir les principaux intéressés :
l'étudiant, le responsable en entreprise et le personnel administratif en charge de gérer les conventions de stage.

	Dans le cas où la fiche n'est pas validée et seulement à compléter, on se propose seulement de prévenir le responsable en entreprise dans 
le but qu'il puisse compléter. Cette notification pourra être accompagné d'une remarque si le professeur référent le juge utile. 

	Dans le dernier cas, il faut réinitialiser la procédure de demande de stage. C'est dans ce but qu'on se propose de simplement supprimer 
les liens entre les bonnes instances de "Student", de "Company" et de "InformationSheet". Un mail sera envoyé dans le but d'informer l'étudiant
qu'il faudra poursuivre en trouvant un autre stage et l'entreprise pour qu'elle sache que le stage ne correspondait pas aux exigeances spécifiées
par l'école. 

\newpage
\begin{figure}[h]
\centering
\includegraphics[width=15cm]{images/validationSeqDiagram.png}
\caption{Diagramme de séquence de la validation de la fiche de renseignement}
\end{figure}

\newpage
\section{Cas du changement d'année}

	Pour ce diagramme de séquence, on suppose que la fonction est appelée par l'utilisateur au moment où les notes des stages
précédant on toutes été établies grâce. Un système de deadlines sera mis en place par les professeurs à leur convenance. 

	Le diagramme de séquence représentant le cas de la transition d'une année scolaire à l'autre, débute à partir du moment
où la classe gérant l'interface graphique appelle la fonction "submisStudentStatus(Data d)". On soppose ici que les données
sont sous forme d'un tableau à deux colonnes. De plus on aura la liste des nouveaux professeur référents, ainsi que la liste des professeurs
responsable des différents stage. La premières est pour le nom de l'élève et la deuxième est pour l'année dans laquelle
il passera. La procédure se répétera pour chaque élève. On commence par savoir si l'élève est nouveau en regardant sa prochaine année.
S'il est nouveau, on lui crée une instance en initialisant ses données (son année, son nom), sinon on retrouve l'instance de "Student" 
correspondant à l'élève, ce qui nous donne accès à son année courrante. 

	Si c'est un quatrième année, celà veut dire qu'on a déjà retiré ses informations de la base de donnée. Sinon, il faut retrouver 
l'entreprise où l'élève à effectué son dernier stage ainsi que l'instance de "InformationSheet" correspondant. De cette manière, on peut 
supprimer le lien entre l'instance de "Company" et l'instance de "InformationSheet" ainsi que réinitialiser les informations de 
"InformationSheet". De ce fait, si un élève d'une année passe à une nouvelle année, on peut supprimer les informations du stage précédant 
dans la base de donnée pour qu'il puisse commencer un nouveau stage. Si un élève redouble une année, on ne supprime pas son stage. Il ne 
devra ainsi pas réaliser un nouveau stage. Pour le passage de la 2ème à le 3ème année, ainsi que pour le passage de la 1ère à la 2ème année, 
on s'assure également d'attribuer à l'élève son nouveau professeur référent qui l'aidera au fur et à mesure du déroulement de son stage. 
Il n'est pas nécessaire d'attribuer de professeur référent en première année pour le stage ouvrier. 

	Enfin, indépendamment du choix des élèves, on met à jour les professeurs responsables des stages de 1A, des stages de 2A ainsi que 
des stages de 3A.  

\newpage
\begin{figure}[h]
\centering
\includegraphics[width=15cm]{images/newYearSeqDiag.png}
\caption{Diagramme de séquence de la transition d'une année scolaire à l'autre}
\end{figure}

\newpage
\chapter{Diagramme de classe}
%\addcontentsline{toc}{chapter}{Diagramme de classe}


%\centering
%\includegraphics[width=15cm]{images/classDiagramme.png}
%\caption{Diagramme de classe}
%\end{figure}



\end{document}
